\documentclass{article}

\usepackage{amsmath,amsfonts}
\usepackage[margin=3cm]{geometry}

\usepackage{multirow}

\usepackage{hyperref}
\hypersetup{
    colorlinks = true,
    linkcolor  = blue,
    urlcolor   = blue,
    citecolor  = blue
}

\usepackage{longtable}

\usepackage[numbers]{natbib}
\defcitealias{Cormen}{Cormen}
\defcitealias{Goodrich}{Goodrich}

\usepackage[usenames, dvipsnames]{xcolor}

\usepackage{forest}

\usepackage{listings}
\lstdefinestyle{pseudo}
{
    keywordstyle = [1]{\normalfont\bfseries},
    keywordstyle = [2]{\normalfont\it},
    keywordstyle = [3]{\normalfont},
    morekeywords = [1]{repeat, for, to, return, if},
    morekeywords = [2]{A, B, N, M, output, accum, count, prev, i, j, k, d, start, end},
    morekeywords = [3]{let},
    morecomment = [l][\color{BrickRed}\it]{//}
}

\title{Solutions for Data Structures and Algorithms Spring 2023 — Problem Sets}
\author{By Dmitriy Okoneshnikov, B22-DSAI-04}

\begin{document}

\maketitle

\section*{Week 8. Problem set}

\begin{enumerate}
    \item Insert the following keys into an initially empty red-black tree [\citetalias{Cormen}, Chapter 13]:

    \[20, 13, 10, 7, 9, 1\]

    Show the state of the tree after each insertion (your answer must contain 6 trees).

    \textbf{Answer.}

    \begin{enumerate}
        \item After inserting $20$:

        \begin{forest}
        for tree = {circle, draw}
        [20, color={black}
            [NIL, color={black}]
            [NIL, color={black}]
        ]
        \end{forest}
        
        \item After inserting $13$:

        \begin{forest}
        for tree = {circle, draw}
        [20, color={black}
            [13, color={red}
                [NIL, color={black}]
                [NIL, color={black}]
            ]
            [NIL, color={black}]
        ]
        \end{forest}
        
        \item After inserting $10$:

        \begin{forest}
        for tree = {circle, draw}
        [13, color={black}
            [10, color={red}
                [NIL, color={black}]
                [NIL, color={black}]
            ]
            [20, color={red}
                [NIL, color={black}]
                [NIL, color={black}]
            ]
        ]
        \end{forest}
        
        \item After inserting $7$:

        \begin{forest}
        for tree = {circle, draw}
        [13, color={black}
            [10, color={black}
                [7, color={red}
                    [NIL, color={black}]
                    [NIL, color={black}]
                ]
                [NIL, color={black}]
            ]
            [20, color={black}
                [NIL, color={black}]
                [NIL, color={black}]
            ]
        ]
        \end{forest}
        
        \item After inserting $9$:

        \begin{forest}
        for tree = {circle, draw}
        [13, color={black}
            [9, color={black}
                [7, color={red}
                    [NIL, color={black}]
                    [NIL, color={black}]
                ]
                [10, color={red}
                    [NIL, color={black}]
                    [NIL, color={black}]
                ]
            ]
            [20, color={black}
                [NIL, color={black}]
                [NIL, color={black}]
            ]
        ]
        \end{forest}
        
        \item After inserting $1$:

        \begin{forest}
        for tree = {circle, draw}
        [9, color={black}
            [7, color={black}
                [1, color={red}
                    [NIL, color={black}]
                    [NIL, color={black}]
                ]
                [NIL, color={black}]
            ]
            [13, color={black}
                [10, color={red}
                    [NIL, color={black}]
                    [NIL, color={black}]
                ]
                [20, color={red}
                    [NIL, color={black}]
                    [NIL, color={black}]
                ]
            ]
        ]
        \end{forest}
    \end{enumerate}
    
    \item Insert the following keys into an initially empty AVL tree [\citetalias{Goodrich}, Section 11.3]:

    \[20, 13, 10, 7, 9, 1\]

    Show the state of the tree after each insertion (your answer must contain 6 trees).

    \textbf{Answer.}

    \begin{enumerate}
        \item After inserting $20$:

        \begin{forest}
        for tree = {circle, draw}
        [20
            [NIL]
            [NIL]
        ]
        \end{forest}
        
        \item After inserting $13$:

        \begin{forest}
        for tree = {circle, draw}
        [20
            [13
                [NIL]
                [NIL]
            ]
            [NIL]
        ]
        \end{forest}
        
        \item After inserting $10$:

        \begin{forest}
        for tree = {circle, draw}
        [13
            [10
                [NIL]
                [NIL]
            ]
            [20
                [NIL]
                [NIL]
            ]
        ]
        \end{forest}
        
        \item After inserting $7$:

        \begin{forest}
        for tree = {circle, draw}
        [13
            [10
                [7
                    [NIL]
                    [NIL]
                ]
                [NIL]
            ]
            [20
                [NIL]
                [NIL]
            ]
        ]
        \end{forest}
        
        \item After inserting $9$:

        \begin{forest}
        for tree = {circle, draw}
        [13
            [9
                [7
                    [NIL]
                    [NIL]
                ]
                [10
                    [NIL]
                    [NIL]
                ]
            ]
            [20
                [NIL]
                [NIL]
            ]
        ]
        \end{forest}
        
        \item After inserting $1$:

        \begin{forest}
        for tree = {circle, draw}
        [9
            [7
                [1
                    [NIL]
                    [NIL]
                ]
                [NIL]
            ]
            [13
                [10
                   [NIL]
                   [NIL]
                ]
                [20
                   [NIL]
                   [NIL]
                ]
            ]
        ]
        \end{forest}
    \end{enumerate}
\end{enumerate}

\begin{thebibliography}{9}
\bibitem{Cormen}
  T. H. Cormen, C. E. Leiserson, R. L. Rivest and C. Stein.
  \textit{Introduction to Algorithms, Fourth Edition.}
  The MIT Press
  2022.
\bibitem{Goodrich}
  M. T. Goodrich, R. Tamassia, and M. H. Goldwasser.
  \textit{Data Structures and Algorithms in Java.}
  WILEY
  2014.
\end{thebibliography}

\end{document}
