\documentclass{article}

\usepackage[utf8]{inputenc}
\usepackage{amsmath,amsfonts}
\usepackage[margin=3cm]{geometry}

\usepackage{multirow}

\usepackage{hyperref}
\hypersetup{
    colorlinks = true,
    linkcolor  = blue,
    urlcolor   = blue,
    citecolor  = blue
}

\usepackage[numbers]{natbib}
\defcitealias{Cormen}{Cormen}

\usepackage[usenames, dvipsnames]{xcolor}

\usepackage{listings}
\lstdefinestyle{pseudo}
{
    keywordstyle = [2]{\color{blue}},
    keywordstyle = [3]{\color{red}},
    keywordstyle = [4]{\color{teal}},
    morekeywords = [2]{secret},
    morekeywords = [3]{for, while},
    morekeywords = [4]{and,exchange,with},
    moredelim    = [s][\color{ForestGreen}\it]{/*}{*/}
}

\title{Solutions for Data Structures and Algorithms Spring 2023 — Problem Sets}
\author{By Dmitriy Okoneshnikov, B22-DSAI-04}
\date{January 30-31, 2023}

\begin{document}

\maketitle

\section*{Week 2. Problem set}

\begin{enumerate}
    \item In [\citetalias{Cormen}, Section 16.1], a total amortised time complexity is computed for a sequence of $n$ \textsc{Increment} operations starting from an initially zero counter. Compute the total amortised time complexity for a sequence of $n$ \textsc{Increment} operations starting from a non-zero counter with value $k$. No justification is required, provide your answer using big-Oh notation.

    \textbf{Answer.} $O(1)$

    \item In [\citetalias{Cormen}, Section 16.1], a stack with an extra operation \textsc{Multipop} is discussed. Provide an example of a sequence of \textsc{Push}, \textsc{Pop}, and \textsc{Multipop} operations on an initially empty stack, such that
    \begin{enumerate}
        \item the actual total cost of the sequence is 5,
        \item the sequence contains one \textsc{Pop}, and one \textsc{Multipop}, and 3 \textsc{Push} operations, in some order,
        \item \textsc{Multipop}($k$) must be used with $k \geq 2$.
    \end{enumerate}
    No justification is required for this exercise.

    \textbf{Answer.} \textsc{Push}, \textsc{Multipop(2)}, \textsc{Push}, \textsc{Push}, \textsc{Pop}
    
    \item Consider \texttt{StackQueue}, an implementation of the Queue ADT using a pair of stacks: a \textit{front stack} and a \textit{rear stack}:
    \begin{enumerate}
        \item A queue is empty when both stacks are empty.
        \item To perform \texttt{offer(e)}, we \texttt{push(e)} into the rear stack.
        \item To perform \texttt{poll()}, we \texttt{pop()} from the front stack if it is not empty. If the front stack is empty, we repeatedly \texttt{pop()} elements from the rear stack and \texttt{push} them onto the front stack, until the rear stack is empty. Finally, we \texttt{pop()} from the front stack, since it is no longer empty.
    \end{enumerate}
    Perform amortised time complexity analysis for a sequence of \texttt{offer(e)} and \texttt{poll()} operations performed on an initially empty \texttt{StackQueue}. You \textbf{must} apply either the accounting method or the potential method.

    Assume that the execution cost (time) of \texttt{push(e)}, \texttt{pop()}, \texttt{isEmpty()} for the underlying stack implementation is 1.

    \textbf{Solution.}
    
    Amortised time will be calculated using the accounting method.
    
    From the problem statement we know that \texttt{offer} $\equiv$ \texttt{push} and \texttt{poll} $\equiv$ \texttt{pop} + \texttt{isEmpty} + \texttt{push}, so we need to find the amortized costs for the stack operations, and then use them to find the amortized costs for \texttt{offer} and \texttt{poll}.
    
    Actual costs of the stack operations (from the problem statement) are the following:
    
    \begin{tabular}{ l l }
        operation & $c_i$\\
        \texttt{push} &  1,\\
        \texttt{pop} &  1,\\
        \texttt{isEmpty} &  1,\\
    \end{tabular}

    Let us assign the following amortized costs:

    \begin{tabular}{ l l }
        operation & $\hat{c_i}$\\
        \texttt{push} &  5,\\
        \texttt{pop} &  0,\\
        \texttt{isEmpty} &  1,\\
    \end{tabular}

    $\sum_{i=0}^n{\hat{c_i}} \geq \sum_{i=0}^n{c_i}$
    
    $5 + 0 + 1 \geq 1 + 1 + 1$

    $6 \geq 3 \Rightarrow$ true

    Let us see how to pay for any sequence of queue operations by charging the
amortized costs. 

    \textbf{Enqueue.}
    \begin{enumerate}
        \item When a new element is enqueued, we use 1 unit of cost for the enqueue and we leave the other 4 units of cost for future. 
    \end{enumerate}
    
    \textbf{Dequeue.} 
    \begin{enumerate}
        \item When an element is dequeued, we use 1 unit of cost to check if the front stack is empty.
        \item Front stack is empty.
        \begin{enumerate}
            \item On each iteration check if rear stack is empty. This costs 1 unit of cost, each element will have 3 units of cost left. 
            \item Pop elements from the rear stack using another 1 unit of cost, each element will have 2 units of cost left.
            \item Push elements into the front stack using 1 unit of cost, each element will have 1 unit of cost left.
            \item Pop the element from the front stack using the last 1 unit of cost that this element has.
        \end{enumerate}
        \item Front stack is not empty. 
        \begin{enumerate}
            \item Pop the element from the front stack using 1 unit of cost.
        \end{enumerate}
    \end{enumerate}

    Therefore, a sequence of $n$ \texttt{offer} operations is $5 \cdot n = O(n)$ and a sequence of $n$ \texttt{poll} operations is $1 \cdot n = O(n)$ (we have to pay only for one \texttt{isEmpty} operation).

\end{enumerate}

\begin{thebibliography}{9}
\bibitem{Cormen}
  T. H. Cormen, C. E. Leiserson, R. L. Rivest and C. Stein.
  \textit{Introduction to Algorithms, Fourth Edition.}
  The MIT Press
  2022
\end{thebibliography}

\end{document}
