\documentclass{article}

\usepackage[utf8]{inputenc}
\usepackage{amsmath,amsfonts}
\usepackage[margin=3cm]{geometry}

\usepackage{multirow}

\usepackage{hyperref}
\hypersetup{
    colorlinks = true,
    linkcolor  = blue,
    urlcolor   = blue,
    citecolor  = blue
}

\usepackage[numbers]{natbib}
\defcitealias{Cormen}{Cormen}

\usepackage[usenames, dvipsnames]{xcolor}

\usepackage{listings}
\definecolor{pblue}{rgb}{0.13,0.13,1}
\definecolor{pgreen}{rgb}{0,0.5,0}
\definecolor{pred}{rgb}{0.9,0,0}
\definecolor{pgrey}{rgb}{0.46,0.45,0.48}

\usepackage{minted}

\title{Solutions for Data Structures and Algorithms Spring 2023 — Problem Sets}
\author{By Dmitriy Okoneshnikov, B22-DSAI-04}
\date{February 18, 2023}

\begin{document}

\maketitle

\section*{Homework 1. Problem set}

\subsection*{Asymptotics (32 points)}
\begin{enumerate}
    \item Prove or disprove the following statements. You must provide a formal proof. You may use the definitions of the asymptotic notations as well as their properties and properties of common functions, \textbf{as long as you properly reference} them (e.g. by specifying the exact place in Cormen where this property is introduced)!

    \begin{enumerate}
        \item $n^3\log{n} = \Omega(3n\log{n})$
        
        \textbf{Solution.}

        By definition [\citetalias{Cormen}, Section 3.2] $f(n) = \Omega(g(n))$ means that $\exists c > 0, n_0 > 0: \forall n > n_0: f(n) \geq cg(n) \geq 0$.
        
        $n^3\log{n} \geq 3cn\log{n}$

        Let $n > n_0 > 1$

        $n^3 \geq 3cn$

        $n^2 \geq 3c$

        $c \leq \frac{n^2}{3}$

        This inequality holds true, e.g., when $c = 2, n = 3$.

        \textbf{Q.E.D.}
        
        \item $n^\frac{9}{2} + n^4\log{n} + n^2 = O(n^4\log{n})$

        \textbf{Solution.}

        By definition [\citetalias{Cormen}, Section 3.2] $f(n) = O(g(n))$ means that $\exists c > 0, n_0 > 0: \forall n > n_0: cg(n) \geq f(n) \geq 0$.

        $n^\frac{9}{2} + n^4\log{n} + n^2 \leq cn^4\log{n}$

        $n^\frac{1}{2} + \log{n} + \frac{1}{n^2} \leq c\log{n}$

        Let $n > n_0 > 1$

        $\frac{n^\frac{1}{2}}{\log{n}} + 1 + \frac{1}{n^2\log{n}} \leq c$

        But $n^\frac{1}{2}$ grows faster that $\log{n}$ [\citetalias{Cormen}, Section 3.3, Equation 3.24]. Therefore, we can't choose any $c$ and $n$ that will hold the inequality true.

        \textbf{The statement is not true.}
        
        \item $6^{n+1} + 6(n+1)! + 24n^{42} = O(n!)$

        \textbf{Solution.}

        By definition [\citetalias{Cormen}, Section 3.2] $f(n) = O(g(n))$ means that $\exists c > 0, n_0 > 0: \forall n > n_0: cg(n) \geq f(n) \geq 0$.

        $6^{n+1} + 6(n+1)! + 24n^{42} \leq cn!$

        $\frac{6^{n+1}}{n!} + 6(n+1) + \frac{24n^{42}}{n!} \leq c$

        But it is obvious that there are no $c, n_0$ that will satisfy the inequality. As for any $c, n_0$ there will be such $n > n_0$ that $6(n+1) > c$.

        \textbf{The statement is not true.}
        
        \item There exists a constant $\varepsilon > 0$ such that $\frac{n}{\log{n}} = O(n^{1-\varepsilon})$

        By definition [\citetalias{Cormen}, Section 3.2] $f(n) = O(g(n))$ means that $\exists c > 0, n_0 > 0: \forall n > n_0: cg(n) \geq f(n) \geq 0$.

        $\frac{n}{\log{n}} \leq cn^{1-\varepsilon}$

        $\frac{n^\varepsilon}{\log{n}} \leq c$

        $n^\varepsilon \leq c\log{n}$

        The latter means $n^\varepsilon = O(\log{n})$, but $n^\varepsilon$ grows faster that $\log{n}$ [\citetalias{Cormen}, Section 3.3, Equation 3.24].

        \textbf{The statement is not true.}
        
    \end{enumerate}

    \item For each of the following recurrences, apply the master theorem yielding a closed form formula. You must specify which case is applied, explicitly check the necessary conditions, and provide final answer using $\Theta$-notation. If the theorem cannot be applied you must provide justification.

    \begin{enumerate}
        \item $T(n) = 2T(\frac{n}{3}) + \log{n}$

        \textbf{Solution.}

        Case \#1 is used [\citetalias{Cormen}, Section 4.5], which states that if $\exists \varepsilon > 0: f(n) = O(n^{\log_b{a-\varepsilon}})$, then $T(n) = \Theta(n^{\log_b{a}})$.

        $log(n) = O(n^{\log_3{2-\varepsilon}})$

        From [\citetalias{Cormen}, Section 3.3, Equation 3.24] we know that $log^b{n} = o(n^\alpha)$, where $\alpha > 0$. Therefore, $log^b{n} = O(n^\alpha)$. Then we need any $\varepsilon$ that will make the logarithm positive. For example, $\varepsilon = 0.5$.

        The case's condition is satisfied, therefore, $T(n) = \Theta(n^{\log_3{2}})$.
        
        \textbf{Answer.} $\Theta(n^{\log_3{2}})$
        
        \item $T(n) = 3T(\frac{n}{9}) + \sqrt{n}$

        \textbf{Solution.}

        Case \#2 is used [\citetalias{Cormen}, Section 4.5], which states that if $\exists k \geq 0: f(n) = \Theta(n^{\log_b{a}}\lg^k{n})$, then $T(n) = \Theta(n^{\log_b{a}}\lg^{k+1}{n})$.
    
        $\sqrt{n} = \Theta(n^\frac{1}{2}\lg^k{n})$
    
        Let $k = 0$
    
        $\sqrt{n} = \Theta(\sqrt{n})$. It is true as $f(n) = \Theta(f(n))$ [\citetalias{Cormen}, Section 3.2]
    
        The case's condition is satisfied, therefore, $T(n) = \Theta(\sqrt{n}\lg{n})$
        
        \textbf{Answer.} $\Theta(\sqrt{n}\lg{n})$
        
        \item $T(n) = 4T(\frac{n}{9}) + \sqrt{n}\log{n}$

        \textbf{Solution.}

        Case \#1 is used [\citetalias{Cormen}, Section 4.5], which states that if $\exists \varepsilon > 0: f(n) = O(n^{\log_b{a-\varepsilon}})$, then $T(n) = \Theta(n^{\log_b{a}})$.

        $\sqrt{n}\log{n} = O(n^{\log_9{4-\varepsilon}})$

        Let $\varepsilon = 0.5$

        $\sqrt{n}\log{n} = O(n^{\log_9{3.5}})$
        
        Using the definition of $O$ [\citetalias{Cormen}, Section 3.2]
        
        $n^{0.5}\log{n} \leq cn^{\log_9{3.5}}$

        $n^{0.5 - \log_9{3.5}}\log{n} \leq c$

        $\frac{1}{n^{\log_9{3.5} - 0.5}}\log{n} \leq c$

        $\log{n} \leq cn^{\log_9{3.5} - 0.5}$

        Due to the definition of $O$ [\citetalias{Cormen}, Section 3.2], the latter means $\log{n} = O(n^{\log_9{3.5} - 0.5})$. But from [\citetalias{Cormen}, Section 3.3, Equation 3.24] we know that $log^b{n} = o(n^\alpha)$, where $\alpha > 0$. Therefore, $log^b{n} = O(n^\alpha)$.

        The case's condition is satisfied, therefore, $T(n) = \Theta(n^{\log_9{4}})$
        
        \textbf{Answer.} $\Theta(n^{\log_9{4}})$
        
        \item $T(n) = 5T(\frac{n}{5}) + \frac{n}{1000}$

        \textbf{Solution.}

        Case \#2 is used [\citetalias{Cormen}, Section 4.5], which states that if $\exists k \geq 0: f(n) = \Theta(n^{\log_b{a}}\lg^k{n})$, then $T(n) = \Theta(n^{\log_b{a}}\lg^{k+1}{n})$.

        $\frac{n}{1000} = \Theta(n^{\log_5{5}}\lg^k{n})$

        $\frac{n}{1000} = \Theta(n\lg^k{n})$

        Let $k = 0$

        $\frac{n}{1000} = \Theta(n)$

        The latter is true as we can choose $c_1 = \frac{1}{1000}$, $c_2 = 1$, $n_0 = 1$ such that $0 \leq c_1 n \leq \frac{n}{1000} \leq c_2 n$ for all $n > n_0$ [\citetalias{Cormen}, Section 3.2], therefore, $0 \leq \frac{1}{1000} \cdot n \leq \frac{n}{1000} \leq 1 \cdot n$.

        The case's condition is satisfied, therefore, $T(n) = \Theta(n\lg{n})$
        
        \textbf{Answer.} $\Theta(n\lg{n})$
        
    \end{enumerate}
\end{enumerate}

\subsection*{Segmented List (18 points)}
Recall the methods of the List ADT:

\begin{minted}{java}
public interface List<E>
{
    int size();
    boolean isEmpty()
    void add(int position, E element);
    E remove (int position);
    E get(int position);
    E set(int position, E element);
}    
\end{minted}

Consider a \texttt{SegmentedList} implementation, that consists of a doubly linked list of arrays (segments) of fixed capacity $k$:
\begin{enumerate}
    \item all segments, except the last one, must contain at least $\lfloor\frac{k}{2}\rfloor$ elements;
    \item when adding at the end of \texttt{SegmentedList}, the element is added in the last segment, if it is not full; otherwise, a new segment of capacity $k$ is added at the end of the linked list;
    \item when adding $x$ at any position, the doubly linked list is scanned from left to right (or right to left) to find the corresponding segment; if the segment is full, it is split into two segments with $\frac{k}{2}$ elements each, and $x$ is inserted into one of them, depending on the insertion position.
    \item when removing an element, only elements in its segment are shifted; if the segment size becomes less than $\lfloor\frac{k}{2}\rfloor$, then we rebalance:
    \begin{enumerate}
        \item if it is the last segment, do nothing;
        \item if it is a middle segment $s_i$ and the sum $m = size(s_{i-1}) + size(s_i) + size(s_{i+1}) < 2k$, then replace these three segments with two segments of size $\frac{m}{2}$;
        \item if it is a middle segment $s_i$ and the sum $m = size(s_{i-1}) + size(s_i) + size(s_{i+1}) \geq 2k$, then replace these three segments with three segments of size $\frac{m}{3}$;
        \item if it is the first segment, and there exist two segments after, then proceed similarly to previous two cases;
        \item if it is the first segment and only one segment after it exists, move as many elements from the last segment to the first one as possible; if the last segment becomes empty, remove it.
    \end{enumerate}
\end{enumerate}

Perform the following analysis for the \texttt{SegmentedList}:
\begin{enumerate}
    \item Argue that the worst case time complexity of \texttt{add(i, e)} is $O(\frac{n}{k} + k)$.

    \textbf{Solution.}

    To add an element to a position, we need to first scan the doubly linked lists to find the segment, it will take in the worst case $O(\frac{n}{k})$, because there are $\frac{n}{k}$ segments. If the segment is full, then it needs to be split into two segments, taking in the worst case $O(k)$, because we need to move the second half of the splitted segment. Therefore, the total worst case time complexity of \texttt{add(i, e)} is $O(\frac{n}{k} + k)$.

    \textbf{Q.E.D}
    
    \item Argue that the worst case time complexity of \texttt{remove(i)} is $O(\frac{n}{k} + k)$.

    \textbf{Solution.}

    To remove an element at a position, we need to first scan the doubly linked lists to find the segment, it will take in the worst case $O(\frac{n}{k})$, because there are $\frac{n}{k}$ segments. Then, we need to shift the elements in the corresponding segment, which takes in the worst case $O(k)$, because we need to move at most $k$ elements. If the size of the segment becomes less than $\lfloor\frac{k}{2}\rfloor$, we need to rebalance, which takes in the worst case $O(k)$. Therefore, the total worst case time complexity of \texttt{remove(i)} is $O(\frac{n}{k} + k)$.

    \textbf{Q.E.D.}
    
    \item What value of $k$ should be chosen in practice? Why?

    \textbf{Answer.}
    
    A small value of $k$ would make adding and removing elements faster as we need to shift fewer elements, but it would increase the number of segments, which would require more memory.
    
    A larger value of $k$ would reduce the number of segments, but it would make adding and removing elements slower as we would need to shift more elements.

    So the optimal value of $k$ can vary depending on the needs and the number of elements.
    
\end{enumerate}

\subsection*{Segmented Queue (+1\% extra credit)}

Consider \texttt{SegmentedList} used as a Queue:
\begin{enumerate}
    \item we enqueue (\texttt{offer(e)}) elements by adding them to the end of the segmented list;
    \item we dequeue (\texttt{poll()}) elements by removing the first element of the segmented list;
    \item in an attempt to make \texttt{remove} more efficient, we do not perform shifts when removing an element in a segment (so there can be a gap on the left of the cells where values are stored); \texttt{add} will create a new segment when it reaches the right end of the last segment, regardless of whether there is empty space to the left;
    \item rebalancing part of \texttt{remove} remains, to keep the invariant that each segment has at least $\lfloor\frac{k}{2}\rfloor$ elements.
\end{enumerate}

Perform amortised analysis for arbitrary sequences of \texttt{offer(e)} and \texttt{poll()} operations applied to an initially empty queue, implemented using \texttt{SegmentedList} in a way described above. Show that amortized cost for any such sequence of length $N$ is $O(N)$ (does not depend on the value of $k$). You \textbf{must} use the accounting method or the potential method.

\textbf{NO SOLUTION.}

\begin{thebibliography}{9}
\bibitem{Cormen}
  T. H. Cormen, C. E. Leiserson, R. L. Rivest and C. Stein.
  \textit{Introduction to Algorithms, Fourth Edition.}
  The MIT Press
  2022
\end{thebibliography}

\end{document}
